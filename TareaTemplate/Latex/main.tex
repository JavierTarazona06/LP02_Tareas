\documentclass{article}
\usepackage{graphicx}
\usepackage[style=ieee]{biblatex} % Establecer el estilo de las referencias como IEEE
\usepackage{xcolor}
\usepackage{hyperref}
\usepackage{titletoc}
\usepackage{adjustbox}

\hypersetup{
    colorlinks=true,
    linkcolor=blue, % Color del texto del enlace
    urlcolor=blue % Color del enlace
}

\usepackage{longtable} % Agrega el paquete longtable

\definecolor{mygreen}{RGB}{0,128,0}

\usepackage{array} % Para personalizar la tabla
\usepackage{booktabs} % Para líneas horizontales de mejor calidad
\usepackage{graphicx} % Paquete para incluir imágenes
\usepackage{float}

% Definir márgenes
\usepackage[margin=1in]{geometry}

\renewcommand{\contentsname}{\textcolor{mygreen}{Tabla de Contenidos}}

\begin{document}

\begin{titlepage}
    \centering
    % Logo de la Universidad
    \includegraphics[width=0.48\textwidth]{logo_universidad.png}
    \par\vspace{2cm}

    % Nombre de la Universidad y detalles del curso
    {\Large \textbf{Universidad Nacional de Colombia} \par}
    \vspace{0.5cm}
    {\large Ingeniería de Sistemas y Computación \par}
    {\large 2025966 Lenguajes de Programación (02)\par}
    \vspace{3cm}

    % Detalles del laboratorio y actividad
    {\large \textbf{Tarea Template} \par}
    {\large Emulación de su computador\par}
    \vspace{3cm}

    % Lista de integrantes
    {\large \textbf{Integrantes:} \par}
    \vspace{0.5cm}
    \begin{tabular}{ll}
    Javier Andrés Tarazona Jiménez & jtarazonaj@unal.edu.co \\
    - & -@unal.edu.co \\
    \end{tabular}
    \par\vspace{3cm}

    % Fecha
    {\large Mayo 7 de 2025 \par}
\end{titlepage}

\tableofcontents % Inserta la tabla de contenidos

\newpage % Salto de página para separar la tabla de contenidos del contenido del documento

% Contenido del artículo----------------------------------------------------------

%---------------------------------------------------------------------------------
% Intro --------------------------------------------------------------------------
%---------------------------------------------------------------------------------

\section{Introducción}\label{sec:intr}

%---------------------------------------------------------------------------------
% Marco Teórico ------------------------------------------------------------------
%---------------------------------------------------------------------------------

\section{Marco Teórico}\label{sec:marc}



\subsection{Contextualización del problema}




%---------------------------------------------------------------------------------
% Descripción y Justificación del Problema a Resolver ----------------------------
%---------------------------------------------------------------------------------

\section{Descripción y Justificación del Problema a Resolver}\label{sec:descr}


\subsection{Objetivo Principal}


%---------------------------------------------------------------------------------
% Diseño de la solución ---------------------------------------------------------
%---------------------------------------------------------------------------------

\section{Diseño de la solución}\label{sec:dis}


\subsection{Metodología}



%---------------------------------------------------------------------------------
% Código Fuente ---------------------------------------------------------
%---------------------------------------------------------------------------------

\section{Código Fuente}\label{sec:cod}

El código fuente completo de este modelo se encuentra adjunto en el buzón 
(11 Tarazona Jimenez Javier Andres 02.zip)
y disponible en el repositorio GitHub del proyecto:

\begin{center}
\url{URL}
\end{center}

El repositorio contiene:
\begin{itemize}
\item A
\item B
\item C
\end{itemize}

%---------------------------------------------------------------------------------
% Manual Usuario ---------------------------------------------------------
%---------------------------------------------------------------------------------

\section{Manual Usuario}\label{sec:man_u}

El primer paso es descargar el archivo \texttt{11 Tarazona Jimenez Javier Andres 02.zip}.

Una vez descargado, descomprímalo y acceda a la carpeta. Dentro de ella, cree un 
entorno virtual utilizando Python 3.12. o superior. Para ello, ejecute el siguiente 
comando en 
la terminal o línea de comandos:

\begin{itemize}
  \item En Windows:
  \begin{verbatim}
    python3.12 -m venv nombre_del_entorno
  \end{verbatim}
  \item En macOS o Linux:
  \begin{verbatim}
    python3.12 -m venv nombre_del_entorno
  \end{verbatim}
\end{itemize}

Donde \texttt{nombre\_del\_entorno} es el nombre que desea asignar a su entorno virtual. 
A continuación, active el entorno virtual:

\begin{itemize}
  \item En Windows:
  \begin{verbatim}
    .\nombre_del_entorno\Scripts\activate
  \end{verbatim}
  \item En macOS o Linux:
  \begin{verbatim}
    source nombre_del_entorno/bin/activate
  \end{verbatim}
\end{itemize}

En el archivo \texttt{constants/program.py} encontrará las constantes del programa. 
En ese archivo, podrá modificar los parámetros de entrada que se detallan más abajo.\\

Después de configurar los parámetros, asegúrese de tener el entorno virtual activado. 
Una vez activo, puede ejecutar el archivo principal con el siguiente comando:

\begin{center}
  \begin{adjustbox}{minipage=\linewidth, center}
  \begin{verbatim}
    python main.py
  \end{verbatim}
  \end{adjustbox}
\end{center}

\textbf{Datos de entrada}

Todos los parámetros de entrada son opcionales.

\begin{itemize}
  \item Número de cursos.
  \item Número máximo de grupos por curso.
  \item Número mínimo de grupos por curso.
  \item Promedio de número de grupos por curso.
  \item Desviación estandar de número de grupos por curso.
  \item Número de docentes, debe ser mayor o igual al número de cursos por maximo de grupos.
  \item Número de aulas, debe ser mayor o igual al número de cursos por maximo de grupos.
  \item Número máximo de cupos por curso.
  \item Número mínimo de cupos por curso.
  \item Promedio de número de cupos por curso.
  \item Desviación estandar de número de cupos por curso.
  \item Número de estudiantes.
  \item Promedio de P.A.P.I de los estudiantes
  \item Desviación estandar de P.A.P.I de los estudiantes
  \item Máximo número de materias que un estudiante va a inscribir (mayor o igual a 3, 
        menor o igual a lista de deseos).
  \item Máximo número de materias deseadas de un estudiante (Predeterminado=9, 
        mayor o igual a 3).
  \item Tiempo de simulación (CONSUMO\_CT)
  \item Iteraciones de las realizaciones (ITERACIONES\_CT)
\end{itemize}


%---------------------------------------------------------------------------------
% Manual Técnico ---------------------------------------------------------
%---------------------------------------------------------------------------------

\section{Manual Técnico}\label{sec:man_t}


\subsection{Fases de la Simulación}


\subsection{Manejo de Datos}

\subsection{Evaluación de la Simulación}


\subsection{Conclusiones y Recomendaciones}


%---------------------------------------------------------------------------------
% Experimentación ---------------------------------------------------------
%---------------------------------------------------------------------------------

\section{Experimentación}\label{sec:exp}

\subsection{Análisis de resultados}

\subsubsection{Escenario 1: }

\subsubsection{Escenario 2: }
 
\subsubsection{Escenario 3: }

\subsubsection{Comparación resultados?}



\section{Referencias}
\renewcommand{\refname}{}

\begin{thebibliography}{9}

\bibitem{ref} \label{ref:vidIntro} A. Párraga, “Cómo se programa un ORDENADOR CUÁNTICO (bien explicado),”
YouTube, 2021. [Online]. Available: \url{https://www.youtube.com/watch?v=eRlQdW1lgJE. }

\bibitem{ref} \label{ref:vidTelepor} Ket.G, “La Teleportación cuántica,” YouTube, 3-Mar-2020. [Online]. 
Available: \url{https://www.youtube.com/watch?v=4BUB41iK25Y.}

\bibitem{ref} \label{ref:Grover} Ket.G, “PROGRAMANDO \#3 Algoritmo de Grover (Qiskit),” YouTube, 3-Mar-2020. 
[Online]. Available: \url{https://www.youtube.com/watch?v=u5Yd0St6MN4.}

\bibitem{ref} \label{ref:IBMQuant} IBM Research, “Running an experiment in the IBM Quantum Experience,” 
YouTube, 4-May-2016. [Online]. 
Available: \url{https://www.youtube.com/watch?v=pYD6bvKLI_c}

\end{thebibliography}

\end{document}