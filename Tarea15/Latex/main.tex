\documentclass{article}
\usepackage{graphicx}
\usepackage[style=ieee]{biblatex} % Establecer el estilo de las referencias como IEEE
\usepackage{xcolor}
\usepackage{hyperref}
\usepackage{titletoc}
\usepackage{adjustbox}

\hypersetup{
    colorlinks=true,
    linkcolor=blue, % Color del texto del enlace
    urlcolor=blue % Color del enlace
}

\usepackage{longtable} % Agrega el paquete longtable

\definecolor{mygreen}{RGB}{0,128,0}

\usepackage{array} % Para personalizar la tabla
\usepackage{booktabs} % Para líneas horizontales de mejor calidad
\usepackage{graphicx} % Paquete para incluir imágenes
\usepackage{float}

% Definir márgenes
\usepackage[margin=1in]{geometry}

\renewcommand{\contentsname}{\textcolor{mygreen}{Tabla de Contenidos}}

\begin{document}

\begin{titlepage}
    \centering
    % Logo de la Universidad
    \includegraphics[width=0.48\textwidth]{logo_universidad.png}
    \par\vspace{2cm}

    % Nombre de la Universidad y detalles del curso
    {\Large \textbf{Universidad Nacional de Colombia} \par}
    \vspace{0.5cm}
    {\large Ingeniería de Sistemas y Computación \par}
    {\large 2025966 Lenguajes de Programación (02)\par}
    \vspace{3cm}

    % Detalles del laboratorio y actividad
    {\large \textbf{Tarea 15} \par}
    {\large Emulación de su computador\par}
    \vspace{3cm}

    % Lista de integrantes
    {\large \textbf{Integrantes:} \par}
    \vspace{0.5cm}
    \begin{tabular}{ll}
    Javier Andrés Tarazona Jiménez & jtarazonaj@unal.edu.co \\
    - & -@unal.edu.co \\
    \end{tabular}
    \par\vspace{3cm}

    % Fecha
    {\large Mayo 7 de 2025 \par}
\end{titlepage}

\tableofcontents % Inserta la tabla de contenidos

\newpage % Salto de página para separar la tabla de contenidos del contenido del documento

% Contenido del artículo----------------------------------------------------------

%---------------------------------------------------------------------------------
% Intro --------------------------------------------------------------------------
%---------------------------------------------------------------------------------

\section{Introducción}\label{sec:intr}

%---------------------------------------------------------------------------------
% Marco Teórico ------------------------------------------------------------------
%---------------------------------------------------------------------------------

\section{Marco Teórico}\label{sec:marc}

\subsection{Contextualización del problema}


%---------------------------------------------------------------------------------
% Descripción y Justificación del Problema a Resolver ----------------------------
%---------------------------------------------------------------------------------

\section{Descripción y Justificación del Problema a Resolver}\label{sec:descr}


\subsection{Objetivo Principal}


%---------------------------------------------------------------------------------
% Diseño de la solución ---------------------------------------------------------
%---------------------------------------------------------------------------------

\section{Diseño de la solución}\label{sec:dis}

\section{Diseño de la solución}\label{sec:dis}

Para la implementación del emulador del computador diseñado por el grupo de trabajo, se propone 
utilizar el paradigma de \textbf{programación orientada a objetos}, aplicando la arquitectura de 
desarrollo \textbf{Modelo-Vista-Controlador (MVC)}. Esto permite separar la lógica funcional 
(modelo), la interfaz de interacción (vista) y la gestión del flujo de ejecución (controlador), 
facilitando la mantenibilidad y escalabilidad del sistema.

\subsection{Modelo}

\subsubsection*{Registros}

Esta clase representa la abstracción de los registros del sistema, incluyendo registros de 
propósito general y registros especiales (PC, SP, IR, ESTADO).

\textbf{Atributos:}
\begin{itemize}
  \item \texttt{memoria} (\texttt{bool[32][64]}): Matriz que representa los 32 registros 
  de 64 bits. Cada fila corresponde a un registro, y cada columna a un bit.
\end{itemize}

\subsubsection*{CPU}

Clase que modela la Unidad Central de Proceso. Coordina la ejecución del ciclo 
\texttt{fetch-decode-execute} y comunica las unidades internas.

\textbf{Atributos:}
\begin{itemize}
  \item \texttt{alu} (\texttt{ALU}): Objeto de la unidad aritmético-lógica.
  \item \texttt{uc} (\texttt{UC}): Objeto de la unidad de control.
  \item \texttt{registros} (\texttt{Registros}): Banco de registros.
  \item \texttt{memoria} (\texttt{Memoria}): Memoria principal.
\end{itemize}

\textbf{Métodos:}
\begin{itemize}
  \item \texttt{\_\_init\_\_(alu, uc)}: Constructor que vincula las unidades de la CPU.
  \item \texttt{fetch()}: Carga en el registro \texttt{IR} la instrucción ubicada en la
      dirección del \texttt{PC}, y actualiza el contador de programa.
  \item \texttt{decode()}: Extrae el \texttt{opcode} y los campos relevantes de 
        la instrucción para preparar su ejecución.
  \item \texttt{execute()}: Llama al método correspondiente definido en la ISA 
        según el \texttt{opcode} decodificado.
\end{itemize}

\subsubsection*{UC (Unidad de Control)}

Esta clase decodifica instrucciones y genera las señales necesarias para su ejecución. 

\textbf{Métodos:}
\begin{itemize}
  \item \texttt{decodificar(ir: str)}: Extrae campos de la instrucción.
  \item \texttt{seleccionar\_accion(opcode: str)}: Invoca la operación correspondiente en la ISA.
\end{itemize}

\subsubsection*{ALU}

Ejecuta operaciones aritméticas y lógicas entre registros o con valores inmediatos.

\textbf{Métodos:}
\begin{itemize}
  \item \texttt{operar(op: str, op1: int, op2: int)}: Devuelve el resultado de aplicar 
        la operación \texttt{op} a los operandos.
  \item \texttt{actualizar\_estado(resultado: int)}: Actualiza los bits del 
        registro \texttt{ESTADO} (C, N, P, D).
\end{itemize}

\subsubsection*{Memoria}

Clase que abstrae la memoria principal, organizada en $2^{24}$ direcciones de 64 bits cada una.

\textbf{Atributos:}
\begin{itemize}
  \item \texttt{arreglo} (\texttt{bool[$2^{24}$][64]}): Arreglo de palabras de memoria.
\end{itemize}

\textbf{Métodos:}
\begin{itemize}
  \item \texttt{leer(direccion: int) → int}: Retorna la palabra almacenada en la dirección dada.
  \item \texttt{escribir(direccion: int, palabra: int)}: Guarda una palabra de 64 bits en 
        la dirección dada.
  \item \texttt{check\_structure(operator: str, direccion: int)}: Verifica si la operación es válida 
        según el segmento de memoria:
    \begin{itemize}
      \item 0: Código
      \item 1: Entrada/Salida
      \item 2: Datos estáticos/globales
      \item 3: Pila
    \end{itemize}
\end{itemize}

\subsubsection*{UnidadES (Entrada/Salida)}

Abstracción de la unidad de E/S. Interactúa con dispositivos externos a través de direcciones 
mapeadas en memoria.

\textbf{Métodos:}
\begin{itemize}
  \item \texttt{leer\_dispositivo(direccion: int)}: Simula lectura desde un periférico.
  \item \texttt{escribir\_dispositivo(direccion: int, valor: int)}: Simula salida de datos hacia 
  un periférico.
\end{itemize}

\subsubsection*{ISA (Instruction Set Architecture)}

Define el conjunto de instrucciones que puede ejecutar el procesador. Cada instrucción tiene 
su propia función que es invocada por la CPU durante la fase de ejecución.

\textbf{Métodos:}
\begin{itemize}
  \item \texttt{SUMA(rd, rb)}, \texttt{MUEV(rd, rb)}, etc.
  \item \texttt{ejecutar(opcode: str, campos: dict)}: Mapea el opcode a la función correspondiente.
\end{itemize}

% -----------------------------------------
% -----------------------------------------

\subsection{Controlador}

El controlador actúa como intermediario entre la vista y el modelo. 
Es responsable de recibir las acciones del usuario, traducirlas en comandos válidos y 
enviarlas a las clases del modelo, así como actualizar la vista con los cambios 
de estado del sistema.

\textbf{Responsabilidades:}
\begin{itemize}
  \item Capturar eventos del usuario (ejecutar paso a paso, correr todo, resetear, cargar programa).
  \item Invocar los métodos del modelo (CPU, memoria, registros, etc.).
  \item Actualizar los datos presentados en la vista según el estado actual del sistema.
\end{itemize}

\textbf{Métodos pensados:}
\begin{itemize}
  \item \texttt{cargar\_programa(ruta)}: Traduce el programa en código máquina y lo 
      almacena en memoria.
  \item \texttt{ejecutar\_paso()}: Ejecuta una instrucción y actualiza la vista.
  \item \texttt{ejecutar\_todo()}: Ejecuta hasta la instrucción \texttt{HALT}.
  \item \texttt{resetear()}: Inicializa todos los componentes del modelo.
\end{itemize}

%------------------------
%------------------------

\subsection{Vista}

La vista representa la interfaz de interacción entre el usuario y el sistema. 
Es implementada como una interfaz gráfica (GUI),
Su propósito es presentar visualmente el estado interno 
del computador (registros, memoria, instrucciones) y permitir la carga de programas, 
control de ejecución y visualización de resultados.

\textbf{Responsabilidades:}
\begin{itemize}
  \item Mostrar en tiempo real el contenido de los registros y de la memoria.
  \item Representar gráficamente en tablas las instrucciones cargadas y el puntero de programa.
  \item Permitir al usuario:
    \begin{itemize}
      \item Cargar un programa ensamblador o en binario.
      \item Iniciar, pausar y reiniciar la ejecución.
      \item Observar los resultados de entrada/salida.
    \end{itemize}
\end{itemize}

\textbf{Ejemplo de componentes de GUI:}
\begin{itemize}
  \item Tabla de registros (R0–R31).
  \item Área de memoria con resaltado por segmento (código, datos, E/S, pila).
  \item Área de código de máquina 
  \item Área de código de máquina relocalizable
  \item Área de código ensamblador
  \item Área de código alto nivel
  \item Consola interactiva para entrada/salida.
\end{itemize}



%---------------------------------------------------------------------------------
% Código Fuente ---------------------------------------------------------
%---------------------------------------------------------------------------------

\section{Código Fuente}\label{sec:cod}

El código fuente completo de este modelo se encuentra adjunto en el buzón 
(11 Tarazona Jimenez Javier Andres 02.zip)
y disponible en el repositorio GitHub del proyecto:

\begin{center}
\url{URL}
\end{center}

El repositorio contiene:
\begin{itemize}
\item A
\item B
\item C
\end{itemize}

%---------------------------------------------------------------------------------
% Manual Usuario ---------------------------------------------------------
%---------------------------------------------------------------------------------

\section{Manual Usuario}\label{sec:man_u}

El primer paso es descargar el archivo \texttt{11 Tarazona Jimenez Javier Andres 02.zip}.

Una vez descargado, descomprímalo y acceda a la carpeta. Dentro de ella, cree un 
entorno virtual utilizando Python 3.12. o superior. Para ello, ejecute el siguiente 
comando en 
la terminal o línea de comandos:

\begin{itemize}
  \item En Windows:
  \begin{verbatim}
    python3.12 -m venv nombre_del_entorno
  \end{verbatim}
  \item En macOS o Linux:
  \begin{verbatim}
    python3.12 -m venv nombre_del_entorno
  \end{verbatim}
\end{itemize}

Donde \texttt{nombre\_del\_entorno} es el nombre que desea asignar a su entorno virtual. 
A continuación, active el entorno virtual:

\begin{itemize}
  \item En Windows:
  \begin{verbatim}
    .\nombre_del_entorno\Scripts\activate
  \end{verbatim}
  \item En macOS o Linux:
  \begin{verbatim}
    source nombre_del_entorno/bin/activate
  \end{verbatim}
\end{itemize}

En el archivo \texttt{constants/program.py} encontrará las constantes del programa. 
En ese archivo, podrá modificar los parámetros de entrada que se detallan más abajo.\\

Después de configurar los parámetros, asegúrese de tener el entorno virtual activado. 
Una vez activo, puede ejecutar el archivo principal con el siguiente comando:

\begin{center}
  \begin{adjustbox}{minipage=\linewidth, center}
  \begin{verbatim}
    python main.py
  \end{verbatim}
  \end{adjustbox}
\end{center}

\textbf{Datos de entrada}

Todos los parámetros de entrada son opcionales.

\begin{itemize}
  \item Número de cursos.
  \item Número máximo de grupos por curso.
  \item Número mínimo de grupos por curso.
  \item Promedio de número de grupos por curso.
  \item Desviación estandar de número de grupos por curso.
  \item Número de docentes, debe ser mayor o igual al número de cursos por maximo de grupos.
  \item Número de aulas, debe ser mayor o igual al número de cursos por maximo de grupos.
  \item Número máximo de cupos por curso.
  \item Número mínimo de cupos por curso.
  \item Promedio de número de cupos por curso.
  \item Desviación estandar de número de cupos por curso.
  \item Número de estudiantes.
  \item Promedio de P.A.P.I de los estudiantes
  \item Desviación estandar de P.A.P.I de los estudiantes
  \item Máximo número de materias que un estudiante va a inscribir (mayor o igual a 3, 
        menor o igual a lista de deseos).
  \item Máximo número de materias deseadas de un estudiante (Predeterminado=9, 
        mayor o igual a 3).
  \item Tiempo de simulación (CONSUMO\_CT)
  \item Iteraciones de las realizaciones (ITERACIONES\_CT)
\end{itemize}


%---------------------------------------------------------------------------------
% Manual Técnico ---------------------------------------------------------
%---------------------------------------------------------------------------------

\section{Manual Técnico}\label{sec:man_t}


\subsection{Fases de la Simulación}


\subsection{Manejo de Datos}

\subsection{Evaluación de la Simulación}


\subsection{Conclusiones y Recomendaciones}


%---------------------------------------------------------------------------------
% Experimentación ---------------------------------------------------------
%---------------------------------------------------------------------------------

\section{Experimentación}\label{sec:exp}

\subsection{Análisis de resultados}

\subsubsection{Escenario 1: }

\subsubsection{Escenario 2: }
 
\subsubsection{Escenario 3: }


\section{Referencias}
\renewcommand{\refname}{}

\begin{thebibliography}{9}

\bibitem{ref} \label{ref:BPS} M. Bichler, S. Merting, and A. Uzunoglu, 
“Assigning Course Schedules: About Preference Elicitation, Fairness, and Truthfulness,” 
arXiv preprint arXiv:1812.02630, 2018. [En línea]. Disponible en: 
\url{https://arxiv.org/abs/1812.02630}


\end{thebibliography}

\end{document}