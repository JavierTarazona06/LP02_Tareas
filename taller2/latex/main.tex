\documentclass{article}
\usepackage{csquotes}
\usepackage[utf8x]{inputenc}
\usepackage[T1,T2A]{fontenc}
\usepackage{fvextra}
\fvset{commandchars=\\\{\}, mathescape=true}
\usepackage{xcolor}
\usepackage{amssymb}       % define \Sha
\usepackage{listings}
\usepackage{listingsutf8}
\usepackage[spanish]{babel}
\usepackage{graphicx}
\usepackage[style=ieee]{biblatex} % Establecer el estilo de las referencias como IEEE
\usepackage{hyperref}
\usepackage{titletoc}
\usepackage{adjustbox}

\definecolor{bg}{rgb}{0.95,0.95,0.95}
\definecolor{gray}{rgb}{0.5,0.5,0.5}
\definecolor{purple}{rgb}{0.58,0,0.82}
\definecolor{bluekeyword}{rgb}{0.26,0.44,0.76}
\definecolor{lightorange}{rgb}{0.8,0.5,0.2}
\definecolor{darkgreen}{rgb}{0.0,0.5,0.0}

\lstdefinestyle{mypython}{
    inputencoding=utf8,           % Indica a listings que use UTF-8
    extendedchars=true,           % Permite caracteres extendidos
    mathescape=true,              % Habilita modo matemático dentro de literales
    language=Python,
    backgroundcolor=\color{bg},
    basicstyle=\ttfamily\footnotesize,
    keywordstyle=\color{bluekeyword}\bfseries,
    commentstyle=\color{gray}\itshape,
    stringstyle=\color{lightorange},
    numberstyle=\tiny\color{gray},
    identifierstyle=\color{black},
    showstringspaces=false,
    numbers=left,
    numbersep=10pt,
    frame=single,
    breaklines=true,
    tabsize=4,
    captionpos=b,
    escapeinside={(*@}{@*)},
    literate=
     {á}{{\'a}}1 {é}{{\'e}}1 {í}{{\'i}}1 {ó}{{\'o}}1 {ú}{{\'u}}1
     {Á}{{\'A}}1 {É}{{\'E}}1 {Í}{{\'I}}1 {Ó}{{\'O}}1 {Ú}{{\'U}}1
     {ñ}{{\~n}}1 {Ñ}{{\~N}}1 {π}{{$\pi$}}1  {Ш}{{\fontencoding{T2A}\selectfont\char"DB}}1
     {¡}{{\textexclamdown}}1 {¿}{{\textquestiondown}}1
}


\hypersetup{
    colorlinks=true,
    linkcolor=blue, % Color del texto del enlace
    urlcolor=blue % Color del enlace
}

\usepackage{longtable} % Agrega el paquete longtable

\definecolor{mygreen}{RGB}{0,128,0}

\usepackage{array} % Para personalizar la tabla
\usepackage{booktabs} % Para líneas horizontales de mejor calidad
\usepackage{graphicx} % Paquete para incluir imágenes
\usepackage{float}
\usepackage[section]{placeins}

% Definir márgenes
\usepackage[margin=1in]{geometry}

\renewcommand{\contentsname}{\textcolor{mygreen}{Tabla de Contenidos}}

\begin{document}

\begin{titlepage}
  \centering
  % Logo de la Universidad
  \includegraphics[width=0.48\textwidth]{logo_universidad.png}
  \par\vspace{2cm}

  % Nombre de la Universidad y detalles del curso
  {\Large \textbf{Universidad Nacional de Colombia} \par}
  \vspace{0.5cm}
  {\large Ingeniería de Sistemas y Computación \par}
  {\large 2025966 Lenguajes de Programación (02)\par}
  \vspace{3cm}

  % Detalles del laboratorio y actividad
  {\large \textbf{Taller 2} \par}
  {\large Analizador Sintáctico y Semántico\par}
  \vspace{3cm}

  % Lista de integrantes
  {\large \textbf{Integrantes:} \par}
  \vspace{0.5cm}
  \begin{tabular}{ll}
    Javier Andrés Tarazona Jiménez   & jtarazonaj@unal.edu.co \\
    David Felipe Marin Rosas         & dmarinro@unal.edu.co   \\
    Juan Sebastian Muñoz Lemus       & jumunozle@unal.edu.co          \\
    Eder José Hernández Buelvas      & ehernandezbu@unal.edu.co          \\
    Axel Gomez Moreno                & axgomezm@unal.edu.co          \\
    Daniel Santiago Delgado Pinilla  & ddelgadopi@unal.edu.co          \\
  \end{tabular}
  \par\vspace{3cm}

  % Fecha
  {\large Junio 24 de 2025 \par}
\end{titlepage}

\tableofcontents % Inserta la tabla de contenidos

\newpage % Salto de página para separar la tabla de contenidos del contenido del documento

% Contenido del artículo----------------------------------------------------------

%---------------------------------------------------------------------------------
% Intro --------------------------------------------------------------------------
%---------------------------------------------------------------------------------

\section{Introducción}\label{sec:intr}


%---------------------------------------------------------------------------------
% Marco Teórico ------------------------------------------------------------------
%---------------------------------------------------------------------------------

\section{Marco Teórico}\label{sec:marc}



%---------------------------------------------------------------------------------
% Descripción y Justificación del Problema a Resolver ----------------------------
%---------------------------------------------------------------------------------

\section{Descripción y Justificación del Problema a Resolver}\label{sec:descr}



%---------------------------------------------------------------------------------
% Diseño de la solución ---------------------------------------------------------
%---------------------------------------------------------------------------------

\section{Diseño de la solución}\label{sec:dis}


%---------------------------------------------------------------------------------
% Código Fuente ---------------------------------------------------------
%---------------------------------------------------------------------------------

\section{Código Fuente}\label{sec:cod}

El código fuente completo de este modelo se encuentra adjunto como 
(taller2.zip)
y disponible en el repositorio GitHub del proyecto:

\begin{center}
\url{https://github.com/JavierTarazona06/LP02_Tareas/tree/main/taller2/code}
\end{center}

%---------------------------------------------------------------------------------
% Manual Usuario ---------------------------------------------------------
%---------------------------------------------------------------------------------

\section{Manual Usuario}\label{sec:man_u}

El primer paso es descargar el archivo \texttt{taller1.zip}.

El repositorio contiene:
\begin{itemize}
  \item \textbf{Código principal}
  \begin{description}
    \item[\texttt{a\_lexico.py}] Código principal del analizador léxico: 
      define la lista de tokens, patrones (regex) y funciones \texttt{t\_…} que definen patrones
      para cada categoría, y construye el lexer con PLY.
  \end{description}

  \item \textbf{Otros}
  \begin{description}
    \item[\texttt{InstallationGuide.md}] Guía paso a paso (Markdown) para crear y 
      activar el entorno virtual y luego instalar las dependencias necesarias con 
      \texttt{pip install -r requirements.txt}.
    \item[\texttt{requirements.txt}] Lista de dependencias del proyecto 
      (principalmente la versión de \texttt{ply} necesaria para ejecutar el analizador).
  \end{description}

  \item \textbf{Testing}
  \begin{description}
    \item[\texttt{tests/prototype1.txt}] Primer caso de prueba...
  \end{description}
\end{itemize}


Una vez descargado, descomprímalo y acceda a la carpeta. Dentro de ella, cree un 
entorno virtual utilizando Python 3.10. o superior. Para ello, ejecute el siguiente 
comando en 
la terminal o línea de comandos:

\begin{itemize}
  \item En Windows:
        \begin{verbatim}
    python3.12 -m venv nombre_del_entorno
  \end{verbatim}
  \item En macOS o Linux:
        \begin{verbatim}
    python3.12 -m venv nombre_del_entorno
  \end{verbatim}
\end{itemize}

Donde \texttt{nombre\_del\_entorno} es el nombre que desea asignar a su entorno virtual.
A continuación, active el entorno virtual:

\begin{itemize}
  \item En Windows:
        \begin{verbatim}
    .\nombre_del_entorno\Scripts\activate
  \end{verbatim}
  \item En macOS o Linux:
        \begin{verbatim}
    source nombre_del_entorno/bin/activate
  \end{verbatim}
\end{itemize}

Asegúrese de tener el entorno virtual activado. 
A continuación, descargue las dependencias necesarias 
(en este caso, únicamente \texttt{PLY} para gestionar \texttt{Lex}):

\begin{verbatim}
    pip install -r requirements.txt
\end{verbatim}

Una vez activo y el entorno instalado, puede ejecutar el archivo 
principal con el siguiente comando:

\begin{center}
  \begin{adjustbox}{minipage=\linewidth, center}
  \begin{verbatim}
    python a_lexico.py
  \end{verbatim}
  \end{adjustbox}
\end{center}


%---------------------------------------------------------------------------------
% Manual Técnico ---------------------------------------------------------
%---------------------------------------------------------------------------------

\section{Manual Técnico}\label{sec:man_t}


\section{Experimentación}\label{sec:exp}

\subsection{Análisis de resultados}

\subsubsection{Escenario 1: Código correcto completo}



\subsubsection{Escenario 2: Código con errores léxicos}


\subsubsection{Escenario 3:  Anidamientos Complejos }



\section{Conclusiones}


\section{Gramáticas}\label{sec:grammar}

A continuación se describen los análisis sintácticos por categorías:

\subsection{Definiciones Generales}

\subsubsection*{Programa}

\begin{verbatim}
program ::= ('COMMENT' | func_declaration)* main_declaration
\end{verbatim}

Define la estructura general de un programa válido en el lenguaje. 
Permite que haya cualquier cantidad de comentarios y llamadas
definiciones de funciones (func\_call) al inicio, 
pero siempre debe finalizar con una llamada a la función principal 
(main\_call).

\subsubsection*{Main Call}

\begin{verbatim}
main_declaration
  ::= 'PALABCLAVE=Func' 
        ( 'PALABCLAVE=Vacio' | 'TIPOA' | 'TIPOB' ) 
        'PALABCLAVE=Principal' block
\end{verbatim}

\begin{itemize}
    \item Detecta la definición \texttt{Func <tipo> Principal <block>}.
    \item Valida cada parte:
    \begin{itemize}
        \item \texttt{Func} al inicio,
        \item un tipo permitido,
        \item nombre \texttt{Principal}.
    \end{itemize}
    \item Genera un nodo AST \texttt{('main', tipo, block)} para uso 
      posterior.
\end{itemize}

\subsubsection*{Function Call}

\begin{verbatim}
func_declaration 
      ::= 'PALABCLAVE=Func' 
          ('PALABCLAVE=Vacio' | 'TIPOA' | 'TIPOB') 
          'ID' block
\end{verbatim}

\begin{itemize}
    \item Comprueba que empiece con \texttt{Func}.
    \item Verifica que el tipo sea válido.
    \item Construye y devuelve el nodo AST con el nombre y contenido del block.
\end{itemize}



\section{Referencias}
\renewcommand{\refname}{}

\begin{thebibliography}{9}

%---------------------------------------------------------------------------------
% Referencias, aunque creo que mejor deberíamos usar un .bib y llamarlas desde ahí, es más facil ---------------------------------------------------------
%---------------------------------------------------------------------------------

\bibitem{ref} \label{ref:lexPy1} J. R. Levine, T. Mason, and D. 
Brown, “Lex \& Yacc,” 2nd ed., O’Reilly \& Associates, 1992.

\bibitem{ref} \label{ref:lexPy2}  D. M. Beazley, “PLY (Python Lex‐Yacc)
Manual,” Version 3.11, 2023. [Online]. Available: https://www.dabeaz.com/ply/.

\bibitem{ref} \label{ref:rachas} J.~E.~Ortiz~Triviño, ``Lenguaje para 
  procesamiento de rachas,'' Documento interno, Universidad Nacional de 
    Colombia, enviado por correo electrónico, 6 de mayo de 2025.

\end{thebibliography}

\end{document}