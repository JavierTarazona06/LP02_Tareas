\documentclass{article}
\usepackage{csquotes}
\usepackage[utf8x]{inputenc}
\usepackage[T1,T2A]{fontenc}
\usepackage{fvextra}
\fvset{commandchars=\\\{\}, mathescape=true}
\usepackage{xcolor}
\usepackage{amssymb}       % define \Sha
\usepackage{listings}
\usepackage{listingsutf8}
\usepackage[spanish]{babel}
\usepackage{graphicx}
\usepackage[style=ieee]{biblatex} % Establecer el estilo de las referencias como IEEE
\usepackage{hyperref}
\usepackage{titletoc}
\usepackage{adjustbox}

\definecolor{bg}{rgb}{0.95,0.95,0.95}
\definecolor{gray}{rgb}{0.5,0.5,0.5}
\definecolor{purple}{rgb}{0.58,0,0.82}
\definecolor{bluekeyword}{rgb}{0.26,0.44,0.76}
\definecolor{lightorange}{rgb}{0.8,0.5,0.2}
\definecolor{darkgreen}{rgb}{0.0,0.5,0.0}

\lstdefinestyle{mypython}{
    inputencoding=utf8,           % Indica a listings que use UTF-8
    extendedchars=true,           % Permite caracteres extendidos
    mathescape=true,              % Habilita modo matemático dentro de literales
    language=Python,
    backgroundcolor=\color{bg},
    basicstyle=\ttfamily\footnotesize,
    keywordstyle=\color{bluekeyword}\bfseries,
    commentstyle=\color{gray}\itshape,
    stringstyle=\color{lightorange},
    numberstyle=\tiny\color{gray},
    identifierstyle=\color{black},
    showstringspaces=false,
    numbers=left,
    numbersep=10pt,
    frame=single,
    breaklines=true,
    tabsize=4,
    captionpos=b,
    escapeinside={(*@}{@*)},
    literate=
     {á}{{\'a}}1 {é}{{\'e}}1 {í}{{\'i}}1 {ó}{{\'o}}1 {ú}{{\'u}}1
     {Á}{{\'A}}1 {É}{{\'E}}1 {Í}{{\'I}}1 {Ó}{{\'O}}1 {Ú}{{\'U}}1
     {ñ}{{\~n}}1 {Ñ}{{\~N}}1 {π}{{$\pi$}}1  {Ш}{{\fontencoding{T2A}\selectfont\char"DB}}1
     {¡}{{\textexclamdown}}1 {¿}{{\textquestiondown}}1
}


\hypersetup{
    colorlinks=true,
    linkcolor=blue, % Color del texto del enlace
    urlcolor=blue % Color del enlace
}

\usepackage{longtable} % Agrega el paquete longtable

\definecolor{mygreen}{RGB}{0,128,0}

\usepackage{array} % Para personalizar la tabla
\usepackage{booktabs} % Para líneas horizontales de mejor calidad
\usepackage{graphicx} % Paquete para incluir imágenes
\usepackage{float}
\usepackage[section]{placeins}

% Definir márgenes
\usepackage[margin=1in]{geometry}

\renewcommand{\contentsname}{\textcolor{mygreen}{Tabla de Contenidos}}

\begin{document}

\begin{titlepage}
  \centering
  % Logo de la Universidad
  \includegraphics[width=0.48\textwidth]{logo_universidad.png}
  \par\vspace{2cm}

  % Nombre de la Universidad y detalles del curso
  {\Large \textbf{Universidad Nacional de Colombia} \par}
  \vspace{0.5cm}
  {\large Ingeniería de Sistemas y Computación \par}
  {\large 2025966 Lenguajes de Programación (02)\par}
  \vspace{3cm}

  % Detalles del laboratorio y actividad
  {\large \textbf{Taller 2} \par}
  {\large Analizador Sintáctico y Semántico\par}
  \vspace{3cm}

  % Lista de integrantes
  {\large \textbf{Integrantes:} \par}
  \vspace{0.5cm}
  \begin{tabular}{ll}
    Javier Andrés Tarazona Jiménez   & jtarazonaj@unal.edu.co \\
    David Felipe Marin Rosas         & dmarinro@unal.edu.co   \\
    Juan Sebastian Muñoz Lemus       & jumunozle@unal.edu.co          \\
    Eder José Hernández Buelvas      & ehernandezbu@unal.edu.co          \\
    Axel Gomez Moreno                & axgomezm@unal.edu.co          \\
    Daniel Santiago Delgado Pinilla  & ddelgadopi@unal.edu.co          \\
  \end{tabular}
  \par\vspace{3cm}

  % Fecha
  {\large Junio 24 de 2025 \par}
\end{titlepage}

\tableofcontents % Inserta la tabla de contenidos

\newpage % Salto de página para separar la tabla de contenidos del contenido del documento

% Contenido del artículo----------------------------------------------------------

%---------------------------------------------------------------------------------
% Intro --------------------------------------------------------------------------
%---------------------------------------------------------------------------------

\section{Introducción}\label{sec:intr}


%---------------------------------------------------------------------------------
% Marco Teórico ------------------------------------------------------------------
%---------------------------------------------------------------------------------

\section{Marco Teórico}\label{sec:marc}



%---------------------------------------------------------------------------------
% Descripción y Justificación del Problema a Resolver ----------------------------
%---------------------------------------------------------------------------------

\section{Descripción y Justificación del Problema a Resolver}\label{sec:descr}



%---------------------------------------------------------------------------------
% Diseño de la solución ---------------------------------------------------------
%---------------------------------------------------------------------------------

\section{Diseño de la solución}\label{sec:dis}


%---------------------------------------------------------------------------------
% Código Fuente ---------------------------------------------------------
%---------------------------------------------------------------------------------

\section{Código Fuente}\label{sec:cod}

El código fuente completo de este modelo se encuentra adjunto como 
(taller2.zip)
y disponible en el repositorio GitHub del proyecto:

\begin{center}
\url{https://github.com/JavierTarazona06/LP02_Tareas/tree/main/taller2/code}
\end{center}

%---------------------------------------------------------------------------------
% Manual Usuario ---------------------------------------------------------
%---------------------------------------------------------------------------------

\section{Manual Usuario}\label{sec:man_u}

\subsection{Clonar el repositorio}

Abra una terminal y ejecute:

\begin{verbatim}
git clone https://github.com/JavierTarazona06/LP02_Tareas.git
cd LP02_Tareas/taller2/code
\end{verbatim}

\subsection{Crear y activar el entorno virtual}

\textbf{En Windows:}
\begin{verbatim}
python -m venv .venv
.venv\Scripts\activate
\end{verbatim}

\textbf{En macOS o Linux:}
\begin{verbatim}
python3 -m venv .venv
source .venv/bin/activate
\end{verbatim}

\subsection{Instalar dependencias}
El proyecto requiere únicamente el paquete \texttt{PLY} (Python Lex-Yacc):

\begin{verbatim}
pip install ply
\end{verbatim}

O si tienes un archivo \texttt{requirements.txt}:
\begin{verbatim}
pip install -r requirements.txt
\end{verbatim}

\subsection{Ejecutar un programa}
Coloca tu archivo fuente (por ejemplo, \texttt{tests/prototype1.txt}) en la carpeta \texttt{code/tests/}.

Para ejecutar el parser e intérprete sobre tu archivo fuente:
\begin{verbatim}
python parser.py tests/prototype1.txt
\end{verbatim}

Si no especificas un archivo, por defecto se usará \texttt{tests/prototype1.txt}.

\subsection{¿Qué hace el comando?}
\begin{itemize}
    \item Analiza el archivo fuente.
    \item Construye el árbol de sintaxis abstracta (AST).
    \item Ejecuta el programa.
    \item Guarda el AST en un archivo binario (\texttt{ast\_guardado.pickle}).
\end{itemize}

\subsection{Ejemplo de archivo fuente}
Crea un archivo \texttt{tests/prototype1.txt} con el siguiente contenido:

\begin{verbatim}
Func Vacio Principal() {
    Entero x = 5;
    imprimir(x);
}
\end{verbatim}

\subsection{Notas adicionales}
\begin{itemize}
    \item Si ves errores de módulos, asegúrate de tener el entorno virtual activado y las dependencias instaladas.
    \item El intérprete imprime los resultados en la consola.
\end{itemize}


%---------------------------------------------------------------------------------
% Manual Técnico ---------------------------------------------------------
%---------------------------------------------------------------------------------

\section{Manual Técnico}\label{sec:man_t}

\subsection{Estructura del Proyecto}

El proyecto está organizado de la siguiente manera:

\begin{itemize}
    \item \textbf{code/} \\ Carpeta principal del código fuente.
    \begin{itemize}
        \item \texttt{lexer.py}: Analizador léxico (tokens y reglas).
        \item \texttt{parser.py}: Analizador sintáctico y punto de entrada principal.
        \item \texttt{ASTInterpreter.py}: Definición de nodos AST y lógica de interpretación/ejecución.
        \item \texttt{TDA.py}: Tipos de datos abstractos usados por el lenguaje.
        \item \texttt{semantic\_analyzer.py}: (Opcional) Análisis semántico adicional.
        \item \texttt{tests/}: Casos de prueba de entrada.
    \end{itemize}
    \item \textbf{latex/} \\ Documentación y archivos LaTeX.
\end{itemize}

\subsection{Archivos Principales}

\begin{description}
    \item[lexer.py] Define los tokens, expresiones regulares y reglas de análisis léxico usando PLY.
    \item[parser.py] Define la gramática, las reglas de producción y la construcción del AST. También contiene el punto de entrada para ejecutar el parser e intérprete.
    \item[ASTInterpreter.py] Implementa los nodos del AST, el entorno de ejecución y la lógica de evaluación de cada nodo (declaraciones, expresiones, ciclos, funciones, etc.).
    \item[TDA.py] Implementa los tipos de datos abstractos (arreglos, conjuntos, diccionarios, matrices, etc.) usados por el lenguaje.
    \item[tests/prototypeX.txt] Archivos de prueba con programas de ejemplo escritos en el lenguaje diseñado.
\end{description}

\subsection{Flujo de Ejecución}

\begin{enumerate}
    \item El usuario ejecuta \texttt{parser.py} con un archivo fuente como argumento.
    \item El archivo fuente es leído y procesado por el lexer y parser (PLY), construyendo el AST.
    \item El nodo raíz del AST es evaluado, lo que inicia la ejecución del programa.
    \item El entorno de ejecución (\texttt{env}) gestiona variables, funciones y el flujo de control.
    \item Los resultados y salidas se imprimen en consola.
\end{enumerate}

\subsection{Extensión y Mantenimiento}

\begin{itemize}
    \item Para agregar nuevas construcciones al lenguaje, modifica \texttt{parser.py} (reglas de gramática) y \texttt{ASTInterpreter.py} (nodos y lógica de evaluación).
    \item Para agregar nuevos tipos de datos, implementa la clase correspondiente en \texttt{TDA.py} y actualiza las reglas de conversión en \texttt{ASTInterpreter.py}.
    \item Para depuración, puedes imprimir el AST generado o el entorno de ejecución en cualquier punto del código.
    \item Los errores sintácticos y semánticos se reportan con mensajes claros en consola.
\end{itemize}

\subsection{Dependencias}

\begin{itemize}
    \item \textbf{Python 3.10+}
    \item \textbf{PLY} (Python Lex-Yacc)
\end{itemize}

\subsection{Notas de Depuración}

\begin{itemize}
    \item Si el parser no reconoce una construcción, revisa la definición de tokens y reglas en \texttt{lexer.py} y \texttt{parser.py}.
    \item Si el AST no se comporta como esperas, imprime los nodos y el entorno en \texttt{ASTInterpreter.py}.
    \item Usa los archivos de prueba en \texttt{tests/} para validar cambios y nuevas funcionalidades.
\end{itemize}


%---------------------------------------------------------------------------------
% Experimentación ---------------------------------------------------------
%---------------------------------------------------------------------------------

\section{Experimentación}\label{sec:exp}

\subsection{Análisis de resultados}

\subsubsection{Escenario 1: Código correcto completo}



\subsubsection{Escenario 2: Código con errores léxicos}


\subsubsection{Escenario 3:  Anidamientos Complejos }



\section{Conclusiones}


\section{Referencias}
\renewcommand{\refname}{}

\begin{thebibliography}{9}

%---------------------------------------------------------------------------------
% Referencias, aunque creo que mejor deberíamos usar un .bib y llamarlas desde ahí, es más facil ---------------------------------------------------------
%---------------------------------------------------------------------------------

\bibitem{ref} \label{ref:lexPy1} J. R. Levine, T. Mason, and D. 
Brown, “Lex \& Yacc,” 2nd ed., O’Reilly \& Associates, 1992.

\bibitem{ref} \label{ref:lexPy2}  D. M. Beazley, “PLY (Python Lex‐Yacc)
Manual,” Version 3.11, 2023. [Online]. Available: https://www.dabeaz.com/ply/.

\bibitem{ref} \label{ref:rachas} J.~E.~Ortiz~Triviño, ``Lenguaje para 
  procesamiento de rachas,'' Documento interno, Universidad Nacional de 
    Colombia, enviado por correo electrónico, 6 de mayo de 2025.

\end{thebibliography}

\end{document}