\documentclass{article}
\usepackage{graphicx}
\usepackage[style=ieee]{biblatex} % Establecer el estilo de las referencias como IEEE
\usepackage{xcolor}
\usepackage{hyperref}
\usepackage{titletoc}
\usepackage{adjustbox}
\usepackage[spanish]{babel}

\hypersetup{
    colorlinks=true,
    linkcolor=blue, % Color del texto del enlace
    urlcolor=blue % Color del enlace
}

\usepackage{longtable} % Agrega el paquete longtable

\definecolor{mygreen}{RGB}{0,128,0}

\usepackage{array} % Para personalizar la tabla
\usepackage{booktabs} % Para líneas horizontales de mejor calidad
\usepackage{graphicx} % Paquete para incluir imágenes
\usepackage{float}
\usepackage[section]{placeins}

% Definir márgenes
\usepackage[margin=1in]{geometry}

\renewcommand{\contentsname}{\textcolor{mygreen}{Tabla de Contenidos}}

\begin{document}

\begin{titlepage}
    \centering
    % Logo de la Universidad
    \includegraphics[width=0.48\textwidth]{logo_universidad.png}
    \par\vspace{2cm}

    % Nombre de la Universidad y detalles del curso
    {\Large \textbf{Universidad Nacional de Colombia} \par}
    \vspace{0.5cm}
    {\large Ingeniería de Sistemas y Computación \par}
    {\large 2025966 Lenguajes de Programación (02)\par}
    \vspace{3cm}

    % Detalles del laboratorio y actividad
    {\large \textbf{Tarea 32} \par}
    {\large AFD obtenido de la expresión regular de números complejos de la Tarea 28\par}
    \vspace{3cm}

    % Lista de integrantes
    {\large \textbf{Integrantes:} \par}
    \vspace{0.5cm}
    \begin{tabular}{ll}
    Javier Andrés Tarazona Jiménez & jtarazonaj@unal.edu.co   \\
    Eder  José Hernández Buelvas   & ehernandezbu@unal.edu.co \\
    Juan Sebastián Muñoz Lemus     & jumunozle@unal.edu.co   \\
    David Felipe Marin Rosas       & dmarinro@unal.edu.co   \\
    \end{tabular}
    \par\vspace{3cm}

    % Fecha
    {\large 24 de Julio de 2025 \par}
\end{titlepage}

\tableofcontents % Inserta la tabla de contenidos

\newpage % Salto de página para separar la tabla de contenidos del contenido del documento

% Contenido del artículo----------------------------------------------------------

%---------------------------------------------------------------------------------
% Intro --------------------------------------------------------------------------
%---------------------------------------------------------------------------------

\section{Introducción}\label{sec:intr}

El presente documento expone el proceso de diseño, construcción e implementación de un Autómata Finito Determinista (AFD) que permite reconocer números reales en una cadena de texto. El trabajo se desarrolla a partir de la expresión regular utilizada en un analizador léxico que forma parte de un compilador básico para operaciones matemáticas. Se describe el modelo teórico, el problema a resolver, la metodología seguida, y se analizan los resultados obtenidos en diferentes escenarios de prueba.

%---------------------------------------------------------------------------------
% Marco Teórico ------------------------------------------------------------------
%---------------------------------------------------------------------------------

\section{Marco Teórico}\label{sec:marc}

Un Autómata Finito es un modelo matemático de máquina de estados utilizado para el reconocimiento de patrones dentro de cadenas de caracteres. Existen dos tipos principales: los Autómatas Finitos No Deterministas (AFN), que pueden incluir transiciones lambda ($\lambda$), y los Autómatas Finitos Deterministas (AFD), que no poseen ambigüedad en sus transiciones.

La expresión regular \texttt{[0-9]+(\.[0-9]+)?} describe la estructura de un número real que puede ser entero o contener una parte decimal. El AFN se construye primero, permitiendo transiciones $\lambda$ para reflejar opcionalidad, y posteriormente se convierte en un AFD para su implementación práctica en un lenguaje de programación como C++.

\section{Descripción y Justificación del Problema a Resolver}

Se requiere un componente capaz de validar si una cadena ingresada por el usuario representa correctamente un número real conforme a la expresión regular definida. Esta funcionalidad es esencial en compiladores, intérpretes y analizadores léxicos, ya que garantiza la correcta identificación de literales numéricos en expresiones matemáticas.




%---------------------------------------------------------------------------------
% Descripción y Justificación del Problema a Resolver ----------------------------
%---------------------------------------------------------------------------------

\section{Descripción y Justificación del Problema a Resolver}\label{sec:descr}

Se requiere un componente capaz de validar si una cadena ingresada por el usuario representa correctamente un número real conforme a la expresión regular definida. Esta funcionalidad es esencial en compiladores, intérpretes y analizadores léxicos, ya que garantiza la correcta identificación de literales numéricos en expresiones matemáticas.

\subsection{Objetivo Principal}

Desarrollar un AFD funcional e implementar su lógica en C++ para verificar la validez de números reales, junto con la experimentación de diferentes casos de prueba que aseguren su correcto funcionamiento.
%---------------------------------------------------------------------------------
% Diseño de la solución ---------------------------------------------------------
%---------------------------------------------------------------------------------

\section{Diseño de la solución}\label{sec:dis}
El diseño parte de la construcción del $\lambda$-AFN correspondiente a la expresión regular, seguido por su transformación a un AFD.

\subsection{Metodología}

Se siguieron los siguientes pasos:
\begin{enumerate}
    \item Análisis de la expresión regular.
    \item Construcción del $\lambda$-AFN.
    \item Transformación del $\lambda$-AFN al AFD.
    \item Programación del AFD en C++.
    \item Pruebas de validación con entradas variadas.
\end{enumerate}


%---------------------------------------------------------------------------------
% Código Fuente ---------------------------------------------------------
%---------------------------------------------------------------------------------

\section{Código Fuente}\label{sec:cod}

El código fuente completo de este modelo se encuentra adjunto en el buzón 
(32 Muñoz Lemus Juan Sebastian 02.zip)
y disponible en el repositorio GitHub del proyecto:

\begin{center}
\url{https://github.com/JavierTarazona06/LP02_Tareas/tree/main}
\end{center}


%---------------------------------------------------------------------------------
% Manual Usuario ---------------------------------------------------------
%---------------------------------------------------------------------------------

\section{Manual Usuario}\label{sec:man_u}

\begin{itemize}
    \item Compile el archivo C++ usando un compilador como \texttt{g++}.
    \item Ejecute el programa.
    \item Ingrese una cadena de texto cuando se solicite.
    \item El programa indicará si la cadena es un número válido o no.
\end{itemize}

%---------------------------------------------------------------------------------
% Experimentación ---------------------------------------------------------
%---------------------------------------------------------------------------------

\section{Experimentación}\label{sec:exp}

\begin{verbatim}
#include <iostream>
#include <string>

using namespace std;

bool esNumero(const string& s) {
    enum Estado { S0, S1, S2, S3, ERROR };
    Estado estado = S0;

    for (char c : s) {
        switch (estado) {
            case S0:
                if (isdigit(c)) estado = S1;
                else return false;
                break;
            case S1:
                if (isdigit(c)) estado = S1;
                else if (c == '.') estado = S2;
                else return false;
                break;
            case S2:
                if (isdigit(c)) estado = S3;
                else return false;
                break;
            case S3:
                if (isdigit(c)) estado = S3;
                else return false;
                break;
            default:
                return false;
        }
    }

    return estado == S1 || estado == S3;
}

int main() {
    string entrada;
    cout << "Ingrese una cadena: ";
    cin >> entrada;

    if (esNumero(entrada)) {
        cout << "Cadena válida.\n";
    } else {
        cout << "Cadena inválida.\n";
    }

    return 0;
}

\end{verbatim}

Se realizaron tres escenarios de prueba con datos de entrada representativos:
\subsection{Análisis de resultados}

\subsubsection{Escenario 1: Entrada: \texttt{12345} }

\textbf{Resultado Esperado:} Cadena válida (número entero).

\subsubsection{Escenario 2: Entrada: \texttt{78.90}}

\textbf{Resultado Esperado:} Cadena válida (número real).

\subsubsection{Escenario 3: Entrada: \texttt{.456}}

\textbf{Resultado Esperado:} Cadena inválida (no cumple con la expresión).

\subsection{Conclusiones}

Los resultados obtenidos coincidieron con lo esperado:
\begin{itemize}
    \item Se identificaron correctamente números enteros.
    \item Se validaron números con parte decimal.
    \item Se rechazaron entradas inválidas como cadenas sin dígito entero antes del punto.
\end{itemize}


\end{document}